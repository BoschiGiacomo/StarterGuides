% ---------------------------------------------------------------------------- %
%                           Intro to DS Starter Guide                          %
% ---------------------------------------------------------------------------- %
%
%
% This document is meant as a starter guide for people approaching MATLAB and
% the course "Introduction to Data Science" for the first time. It is intended
% to offer guidance on the first step while approaching the subject. 
% ---------------------------------------------------------------------------- %
% Covered topics: - MATLAB installation 
%                 - Buying the textbook 
%                 - Github, account creation and repository of the book 
%                 - Youtube Channel where the lessons are published
%
% --------------------------------- Preamble --------------------------------- %
\documentclass[a4paper, 12pt]{article}
\usepackage{graphicx}
\usepackage[english]{babel}
\usepackage{url}
\usepackage{microtype}
\usepackage[indent=0pt]{parskip} %par skips a line and remove paragraphs indent
\usepackage{hyperref}
\hypersetup{colorlinks=true, 
            linkcolor=blue, 
            citecolor=blue, 
            urlcolor=blue
            }
\usepackage{caption}
\usepackage{subcaption}
\captionsetup[figure]{labelformat=empty}
% \captionsetup[subfigure]{labelformat=empty}
\usepackage{csquotes}
\usepackage{listings}

\lstset{basicstyle=\normalsize\ttfamily,breaklines=false}
%\lstset{framextopmargin=50pt,frame=bottomline}

% \usepackage{mcode}
% ---------------------------------------------------------------------------- %
% title information
\title{Introduction to Data Science\\
\large{Starter Guide}}
\date{}
\author{}
% table of contents
\addto\captionsenglish{\renewcommand{\contentsname}{Contents of this guide:}}
% directory of the images
\graphicspath{{Images/}}
% ----------------------------- start of document ---------------------------- %
\begin{document}
\maketitle % Title
% introduction
\begin{center}
\begin{minipage}{0.95\textwidth}
\textbf{Need help?} \hfill \break
If you encounter any difficulties with installation or course materials,
remember that support is available. You can always ask questions and receive
help from your classmates and the professor via the GitHub Issues page described
later in this guide. \hfill
\end{minipage}
\end{center}

%Table of Contents
\tableofcontents

% Start of document
\newpage
\section{MATLAB license and installation}
First of all, to install MATLAB you will need to have a valid MATLAB license; if
you already have the University email, you don't need to worry about this, and
can skip to the section titled
\hyperref[sec:Installation]{\enquote{Installation}} below.

\subsection{Creating a MATLAB account}
Since you don't have your university email yet. You will need to create a
MathWorks account with your personal email, to exercise on MATLAB, while you
wait to receive the University email.

\begin{enumerate}
	\item Go on \url{https://www.mathworks.com/} and click on
	      \href{https://login.mathworks.com/embedded-login/landing.html?cid=getmatlab&s_tid=gn_getml}
	      {\enquote{Get MATLAB}} banner at the top of the page.
	      	      	      	      
	      \includegraphics[trim= 15cm 2cm 0 0, clip, width=.9\textwidth]{MatAcc1.png}
	      	      	      	      
	\item Click on \textbf{Create account}, and insert your personal email.\hfill \break
	      Sometimes you will be asked to make sure that this is your
	      organization's email, ignore that and select \enquote{continue anyway}
	      	      	      	      
	      \begin{minipage}{0.46\textwidth}
	      	\centering
	      	\includegraphics[width=\textwidth]{MatAcc2.png}
	      \end{minipage}%
	      \begin{minipage}{0.46\textwidth}
	      	\centering
	      	\includegraphics[trim= 0 0 0 2cm, clip, width=\textwidth]{MatAcc3.png}
	      \end{minipage}
	      	      
	\item Create a password for your account
	\item You will now be asked to verify your email. An email containing
	      a verification code will be sent to the email address you used to create the
	      account. Just copy the code inside the email and paste it in the box
	      as required. If you are not getting the email, check in the spam
	      folder.
		  \newpage

	\item Next, you will need to insert some personal information to
	      complete your account. You will also be asked if you are a student and
	      which department are you from. This is how i suggest to fill those
	      boxes:
	      \begin{itemize}
	      	\item Which best describes you? $\rightarrow$ Student
	      	\item Department $\rightarrow$ Business, Economics and Finance
	      	\item Which best describes your role? $\rightarrow$ Student (Graduate-level)
	      \end{itemize}
	\item Congratulations, your account should be all set!\hfill \break
	      Proceed to the next part of the guide to get your MATLAB trial license!
\end{enumerate}

\subsection{Getting a MATLAB license}
If you are still waiting to receive your University email, then you will have to
ask for a trial version of 30 days to be able to exercise while you wait. In
this section we will cover how to get a Free 30 day trial version of MATLAB.\par

\begin{enumerate}
	\item First things first, to get your 30 days trial, you need to visit the official
	      MathWorks website, at \href{https://www.mathworks.com/}{www.mathworks.com} and scroll down until you
	      see the free trial banner.
	      
	      \includegraphics[trim=0 5cm 0 3cm,clip,width=.9\textwidth]{MathworksTrial_Marked.png}
	      
	\item After clicking the \enquote{Start now} box, you will be brought to a page that looks
	      like this:
	      
	      \includegraphics[trim= 5cm 0 0 0, clip, width=.9\textwidth]{MathworksTrial2_Marked.png}
	      \newpage

	\item You will now be asked to fill some personal details, here is some
	      guidance on how to fill some of the fields:
	      
	      \begin{itemize}
	      	\item \textbf{\enquote{Is this request on behalf of a faculty member or research
	      	      advisor?}} $\rightarrow$ Check \enquote{No}
	      	\item \textbf{\enquote{How will you use your trial?}} $\rightarrow$ MATLAB Essentials
	      \end{itemize}
	\item Your license should be activated! Proceed to the
	      \hyperref[sec:Installation]{installation} section
\end{enumerate}

\subsection{Installation}
\label{sec:Installation}
Now that you have a MATLAB License, you are ready to install. 
\begin{enumerate}
	\item Head to
	      \href{https://www.mathworks.com/downloads/}{mathworks.com/downloads}, and
	      select the version you want to download, download the latest one unless you
	      have specific needs. An installer will be downloaded on your PC.
	      	      
	\item Now find the installer (usually located inside your downloads folder),
	      and run it. This will start the guided installation process.
	      	      
	\item You will be asked to select a license (you probably won't have more
	      than one)
	      	      
	\item Select a folder for the installation, if you are unsure, just leave
	      the default one.
	      	      
	\item Select what products do you want to install; for this course you will
	      need to install the following:
	      \begin{itemize}
	      	\item MATLAB
	      	\item Datafeed Toolbox
	      	\item Econometrics Toolbox
	      	\item Financial Toolbox
	      	\item Image Processing Toolbox
	      	\item Mapping Toolbox
		\item Optimization Toolbox		
	      	\item Parallel Computing Toolbox
	      	\item Risk Management Toolbox
	      	\item Statistics and Machine Learning Toolbox
	      \end{itemize}
	      	      
	\item Select additional options as you please, and begin the installation.
	      After installing MATLAB proceed to the next section to install FSDA Toolbox.
\end{enumerate}


\subsection{Installing FSDA Toolbox}
Almost there! You just need to install one last toolbox!

\begin{enumerate}
	\item In MATLAB look at the top bar, find the \textbf{Add-Ons} button and
	      click on it
	      
	      \includegraphics[trim= 0 0 10cm 0, clip, width=.9\textwidth]{Addon 1.png}
	      
	\item The addon explorer will open up. Look for \textbf{FSDA} in the
	      search bar. The first result is the one you want, click on it!
	      
	\item Click on add, the toolbox will begin installation.
	      	
	\item The FSDA toolbox is a very active project and gets updated often, so
	      every once in a while, remember to use the \texttt{tuna} function to ensure
	      that you have the latest release of the toolbox installed.\par
	      To use the function, just write \texttt{tuna} in the command window.
	      
\end{enumerate}
\section{Recordings of lessons}
All the recordings of the course lessons will be published on Youtube, at the
following link:
\href{https://www.youtube.com/@marcorianiIntroDS}{youtube.com/@marcorianiIntroDS}\par
The laboratory exercises will not be recorded.

\section{Accessing the Textbook}

The textbook for this course is \textit{Data Science with MATLAB}, published by
Giappichelli Editore. It is available \textbf{only in ebook format} on the
publisher's website.

Before purchasing, you need to create a Giappichelli account to access the book
after purchase; it will be stored in your personal library on the Giappichelli
MyHome platform.

To get the book, follow these steps:

\begin{enumerate}
	\item Visit the Giappichelli website, at
	      \href{https://www.giappichelli.it/}{www.giappichelli.it} and register
	      an account.
	          
	\item Search for \enquote{Data Science with MATLAB}. \textbf{Be careful: two very
	similar books appear!}
	    
	\item Make sure to buy the one titled \textbf{\enquote{Data Science \emph{with}
	MATLAB}} as this is the English edition required for the course.
	    
	\item \textbf{\underline{Do NOT} buy \enquote{Data Science \emph{con} MATLAB}} as that is the
	      Italian edition.
	          
	\item After purchase, you can now access your ebook from your personal
	library at
	\href{https://myhome.giappichelli.it/dashboard/}{myhome.giappichelli.it/dashboard}.
	      	
\end{enumerate}

\begin{figure}[!h]
	\centering
	\includegraphics[trim= 0 5cm 0 0, clip, width=\textwidth]{Books_marked.png}
	\caption{Make sure you select the correct book!}
\end{figure}

\section{GitHub and additional supporting material}
If you ever have difficulties with course material, MATLAB setup, or using the
textbook resources, don't hesitate to ask for help! You can post questions and
get support from your fellow students and the professor directly in the
repository via the \enquote{Issues} section. Everyone is encouraged to participate and
support each other.
\subsection{GitHub}
GitHub has become over the years the main platform where programmers share their
projects, collaborate and communicate with each other.\par

It not only acts as a place where find and download software, but also discuss,
and collaborate on those projects together.\par

A lot of the material of the book will be available at the following repository:
\href{https://github.com/UniprJRC/DSwithMATLAB}{UniprJRC/DSwithMATLAB}. Here you
will be able to ask questions, interact with other students and the professor
through the GitHub repository \enquote{Issues} section, take part in challenges posted by the
professor, and report mistakes or typos in the book.\par

It is highly advised that you create a GitHub account, so that you will be able
to participate in discussions.\par

\subsection{Downloading textbook resources}
All the files that you need are in the GitHub repository, we will now see how to
clone them into your local MATLAB folder. This process will copy all the files
present on GitHub in a folder on your PC.\par
\begin{enumerate}
	\item First of all you will need to install Git to be able to type Git commands into
	      the MATLAB command window.
	      
	\item Go to \href{https://git-scm.com/install/}{git-scm.com/install}, download the
	      version corresponding to your OS, and follow the instructions.
	      
	\item 
	      Once you have installed Git, you will need to open MATLAB and type in the
	      command window:
	      
\begin{lstlisting}[]
!git clone https://github.com/UniprJRC/DSwithMATLAB
\end{lstlisting}
	      
	      \includegraphics[width=.9\textwidth]{DSwithMATLABclone.png}

		  If executing this command produces the following error:\par
		  \texttt{fatal: could not create work tree dir 'DSwithMATLAB':\newline permission denied}\par
		  It means that you are trying to clone the repository into a folder
		  where you don't have writing permissions. Simply change the root
		  folder and execute the command again.

		  Another possible cause of the cloning process failing, might be that
		  you are cloning it into a folder that is automatically syncronized
		  with the cloud (for example by using Google Drive or OneDrive), it is
		  not advised to clone into such folders because this might interfere
		  with the cloning process.
	      
	\item Now you will see that a folder called \texttt{DSwithMATLAB} has been
	      created in the root of your MATLAB folder, inside it you will be able to find
	      all the contents of the textbook repository.
	      
\end{enumerate}

\subsection{Keeping the Project \enquote{Alive}}

Software has a lifecycle, and without ongoing maintenance and use, interest
fades. We therefore encourage readers to report errors and suggest improvements
via the book's GitHub page:
\href{https://github.com/UniprJRC/DSwithMATLAB}{UniprJRC/DSwithMATLAB}\par
Please file bug reports or patch proposals in the \enquote{Issues} section. We
will respond promptly and, if need be, open a direct communication channel with
the contributor. We thank the more than 1,000 users who have interacted via the
Issues in the first two editions of the Italian version of the book.

\end{document}
